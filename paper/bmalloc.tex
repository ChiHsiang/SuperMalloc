Key insights:
\begin{enumerate}
 \item  Why does jemalloc sometimes stall for a long time?  
 \item why do transactions fail (too big, conflicts with another thread, time slice interupts, page faults)?  We can avoid almost all of that for malloc.


 \item Virtual address space is cheap.  

 We can waste virtual address space.

 In fact, it's problematic to give back address space to some
 operating systems, since the search data structures are not always
 good.

 \item Per-CPU caching gives most of the value of Per-thread caching.

  No point in caching more than fits in L2 cache (L3 cache is shared.  Maybe L3 is worth it on a multi-socket machine), since we'll incur cache missues to use the data.  (See the rant about threadtest below: it doesn't touch the memory.)

 \item RSS is very important for long-running applications.

 \item Hoard's benchmarks measure the wrong thing.
   \begin{enumerate}
   \item cache-scratch is wrong: it measures false sharing.  There's a
     fundamental tradeoff between minimizing the space allocated and
     avoiding false sharing.  We argue that the right decision here is
     to allow the programmer who is using malloc to make this
     decision.  (JE Malloc also agrees.  They say, if you don't want
     false sharing, ask for cache-aligned data.
     \cite{http://people.freebsd.org/~jasone/jemalloc/bsdcan2006/jemalloc.pdf})
   
  \item cache-thrash

  \item larson (not as tough as my server version)

  \item shbench

   \item threadtest is wrong: it doesn't touch the allocated memory.  (There's no point in caching more per thread than fits in cache)

  \item linux-scalability

   \end{enumerate}

  \item Jemalloc benchmarks
  \begin{enumuerate}
 \item malloc-test
 \item super-smack
 \item cca
 \item cfrack
 \item gs
 \item sh6bench
 \item smlng
  \end{enumuerate}


 \item jemalloc has gone for fewer spacings over time.  We went with
   similar spacing.  There's one anomaly near 4K where we are only 2x
   space competitive, otherwise we are 25\% space competetive.
\end{enumerate}
